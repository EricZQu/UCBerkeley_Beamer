\documentclass[pdf,serif]{beamer}
\usepackage[T1]{fontenc}
% \usepackage{lmodern}
\usepackage{anyfontsize}
\usepackage{concmath} 
\usefonttheme{professionalfonts}

% A great summary of beamer fonts: https://www.logicmatters.net/resources/pdfs/MathFonts.pdf

\usepackage{amssymb,amsmath,amsthm,enumerate}
\usepackage[utf8]{inputenc}
\usepackage{array}
\usepackage[parfill]{parskip}
\usepackage{graphicx}
\usepackage{caption}
\captionsetup[figure]{labelformat=empty}
\usepackage{subcaption}
\usepackage{amsmath}
\usepackage{bm}
\usepackage{amsfonts,amscd}
%\usepackage{gensymb}
\usepackage[]{units}
\usepackage{listings}
\usepackage{multicol}
\usepackage{tcolorbox}
\usepackage{physics}

\usepackage{minted}

% \usepackage{fontspec}
% \setromanfont{Georgia}
% \setmainfont{Helvetica Neue}
%\setsansfont[Scale=MatchLowercase]{Open Sans}

\usepackage{natbib}

%new commands
\newcommand{\empy}[1]{{\color{CaliforniaGold}\emph{#1}}} 
\newcommand{\empr}[1]{{\color{BerkeleyBlue}\emph{#1}}}


% \theoremstyle{remark}
% \newtheorem*{remark}{Remark}
% \theoremstyle{definition}

\newcommand{\examplebox}[2]{
\begin{tcolorbox}[colframe=CaliforniaGold,colback=boxgray,title=#1]
#2
\end{tcolorbox}}

% \newcommand{\eld}[1]{\frac{d}{dt}(\frac{\partial L}{\partial \dot #1}) - \frac{\partial L}{\partial #1}=0}
% \newcommand{\euler}[1]{\frac{\partial L}{\partial #1}-\frac{d}{dt}(\frac{\partial L}{\partial \dot #1})}
% \newcommand{\eulerg}[1]{\frac{\partial g}{\partial #1}-\frac{d}{dt}(\frac{\partial g}{\partial \dot #1})}
% \newcommand{\divg}[1]{\nabla\cdot #1}
% \newcommand{\prob}[1]{P(#1\vert I)}

% MATH

\renewenvironment{proof}{{\large {\color{BerkeleyBlue}\textit{Proof.}}}}{\hfill$\square$}
\tcbuselibrary{theorems}
% The numbering is consistent. Remove `use counder from=defi' to disabled
% If you want numbering with sections, add `number within=section' option
\newtcbtheorem{defi}{Definition}%
{colframe=BerkeleyBlue, colback=boxgray, fonttitle=\bfseries}{df}
\newtcbtheorem[use counter from=defi]{lem}{Lemma}%
{colframe=CaliforniaGold, colback=boxgray, fonttitle=\bfseries}{le}
\newtcbtheorem[use counter from=defi]{thm}{Theorem}%
{colframe=BerkeleyBlue, colback=boxgray, fonttitle=\bfseries}{th}
\newtcbtheorem[use counter from=defi]{rem}{Remark}%
{colframe=CaliforniaGold, colback=boxgray, fonttitle=\bfseries}{re}


\usetheme{Berkeley} 
\input{./style_files_Berkeley/my_beamer_defs.sty}
\logo{\includegraphics[height=0.25in]{./style_files_Berkeley/Berkeley-Logo-White.pdf}
\hspace*{0.05em}
}



\title[Buzz Words, maybe Clever Abbreviations]{Some Cool Sounding Buzzwords}
\subtitle{Less Buzzy and Somewhat More Explanaotry Subtitle}


\beamertemplatenavigationsymbolsempty

\begin{document}



\author[Eric Qu, UC Berkeley]{
	\begin{tabular}{c} 
	\Large
	Eric Qu\\
    \footnotesize \href{mailto:ericqu@berkeley.edu}{ericqu@berkeley.edu}
\end{tabular}
\vspace{-3ex}}

\institute{
	\includegraphics[height=0.6in]{./style_files_Berkeley/Berkeley-Seal-WhiteBG.pdf}\\
	Department of Electrical Engineering and Computer Sciences\\
	Berkeley Artificial Intelligence Research Lab\\
	University of California, Berkeley}
\date{\today}

\begin{noheadline}
\begin{frame}\maketitle\end{frame}
\end{noheadline}



\begin{frame}{Goals}{Subtitle}
The main goals for this slide:
\begin{itemize}
	\item This is just to show you how this \empy{template} works
	\item There are two `emphasize' functions used to highlight \& italicize texts
	\begin{itemize}
		\item `\textbackslash empy' does \empy{this} and `\textbackslash empr' does \empr{this}.
	\end{itemize}
	\item The colors in this template is taken from the UC Berkeley brand guide: {\url{https://brand.berkeley.edu/colors/}}
\end{itemize}
\end{frame}



\begin{frame}{Example}
A (hopefully) useful function in this \LaTeX~template is:
\begin{center}
	\textbackslash examplebox\{ExampleTitle\}\{ExampleContents\}
\end{center}
which does this:
\examplebox{Example of the Command \textbackslash examplebox}{
This is what it does. Pretty self-explanatory, isn't it?

Given the color them, I \empr{recommend} using \textbackslash empr inside of examplebox. The \textbackslash empy command does not look \empy{that} good.
}
\end{frame}

\section{Math}
\begin{frame}{Math}
There are color boxes for Definition, Theorem, and Lemma:
\begin{defi}{Test}{te}
\end{defi}
\begin{thm}{Test}{te}
\end{thm}
\begin{lem}{Test}{te}
\end{lem}
\begin{proof}
\end{proof}

You can refer to them by Def. \ref{df:te}, Thm. \ref{th:te}, Lem. \ref{le:te}. 
\end{frame}

\begin{frame}{Math}
	Test some equations:
	$$
	\begin{aligned}
		&\int_{-\infty}^{\infty} \exp \left(a x^4+b x^3+c x^2+d x+f\right) d x\\
		=&\ e^f \sum_{n, m, p=0}^{\infty} \frac{b^{4 n}}{(4 n) !} \frac{c^{2 m}}{(2 m) !} \frac{d^{4 p}}{(4 p) !} \frac{\Gamma\left(3 n+m+p+\frac{1}{4}\right)}{a^{3 n+m+p+\frac{1}{4}}}
	\end{aligned}
	$$

	$$
	p(R, \phi) \sim \int_{-\infty}^{\infty} \frac{\tilde{W}_n(\gamma) \exp \left[\imath R / a\left(\sqrt{k^2 a^2-\gamma^2} \cos \phi\right)\right]}{\left(k^2 a^2-\gamma^2\right)^{3 / 4} H_n^{\prime(1)}\left(\sqrt{k^2 a^2-\gamma^2}\right)} d \gamma
	$$
\end{frame}

\section{Code}

\begin{frame}[fragile]{Code}
	%set the langauge in these brackets after {minted}{LANGUAGE SHORTHAND} 

Test code block:

\begin{minted}{c}
int main() {
	printf("hello, world");
	return 0;
}
\end{minted}

Test inline code: \mintinline{python}{print("hello, world")}
\end{frame}

\section{Citation} 

\begin{frame}{Citation}
	Test Citation: \citep{qu2023data}, \citet{qu2023data}
\end{frame}

\begin{frame}[t, allowframebreaks]
	\frametitle{References}
	\bibliographystyle{apalike}
	\scriptsize
	\bibliography{ref.bib}
\end{frame}

\end{document}